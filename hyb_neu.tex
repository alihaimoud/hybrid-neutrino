%For arxiv submission
\documentclass[useAMS, usenatbib]{mnras}
\usepackage{graphicx,amsmath,color,amssymb}
%\voffset=-0.8in
%MNRAS
%\documentclass[useAMS, usenatbib, usegraphicx, twocolumn]{mnras}

\usepackage[pdftitle={A hybrid particle-analytic method for non-linear neutrino structure}]{hyperref}
% \newcommand{\eprint}[1]{\href{http://arxiv.org/abs/#1}{#1}}
% \newcommand{\adsurl}[1]{\href{#1}{ADS}}

\topmargin -1.5cm
\bibliographystyle{mnras}

\newcommand{\beq}{\begin{equation}}
\newcommand{\eeq}{\end{equation}}
\newcommand{\barr}{\begin{eqnarray}}
\newcommand{\earr}{\end{eqnarray}}

\newcommand{\rme}{\textrm{e}}
\newcommand{\rmH}{\textrm{H}}
\newcommand{\Ly}{\textrm{Ly}}
\newcommand{\pabn}{p_{\textrm{ab}}^n}
\newcommand{\pscn}{p_{\textrm{sc}}^n}
\newcommand{\rmd}{\textrm{d}}
\newcommand{\N}{\mathcal{N}}
\newcommand{\nuc}{\nu_{\rm c}}
\newcommand{\Tm}{T_{\rm m}}
\newcommand{\Tr}{T_{\rm r}}
\newcommand{\nh}{n_{\rm H}}
\newcommand{\bfA}{\boldsymbol{A}}
\newcommand{\bfr}{\boldsymbol{r}}
\newcommand{\bfV}{\boldsymbol{V}}
\newcommand{\bs}{\mathbf}
\newcommand{\mH}{\mathcal{H}}

\newcommand{\natu}{Nature (London)}
\newcommand{\aas}{Bull. Am. Astron. Soc.}
\newcommand{\gadget}{{\small GADGET\,}}

\newcommand{\spb}[1]{{\textsc{\textcolor{red}{[{\bf SPB}: #1]}}}}
\newcommand{\yah}[1]{{\textcolor{blue}{[{\bf YAH}: #1]}}}



\newcommand{\Mpch}{\,\mathrm{Mpc} \,h^{-1}}
\newcommand{\hMpc}{h^{-1}\,\mathrm{Mpc}}
\newcommand{\Lya}{Lyman-$\alpha\;$}
%%%%%%%%%%%%%%%%%%%%%%%%%%%%%%%%%%%%%%%%%%%%%%%%%%%%%%%%%%%%%%%%%%%%%%%%%%%%%%%%%%%%%%%%%%%%%%%%%%


\title{A Hybrid Particle-Analytic Method for Non-LInear Neutrino Structure}
\author[ S. Bird and Y. Ali-Ha\"{\i}moud]{
  Simeon Bird\thanks{E-mail: spb@ias.edu} and Yacine Ali-Ha\"{\i}moud\thanks{E-mail: yacine@ias.edu}\vspace{1.5mm}\\
No longer at Institute for Advanced Study, Einstein Drive, Princeton, New Jersey 08540}

\begin{document}

\date{\today}

\pagerange{\pageref{firstpage}--\pageref{lastpage}} \pubyear{2012}
\pagenumbering{arabic}
\label{firstpage}

\maketitle

\begin{abstract}
\end{abstract}

\begin{keywords}
        neutrinos - cosmology: large-scale structure of Universe - cosmology: dark matter
\end{keywords}

\section{Introduction}

Our paper is the greatest paper in the world. Blah de blah.
This paper is organised as follows. In Section \ref{sec:methods},
we describe our simulations methods. We describe our
results and compare them against other methods
in Section \ref{sec:results}. We conclude in Section \ref{sec:conclusion}.
Appendix \ref{sec:manual} is a users manual for our neutrino module.
\cite{AHB}

\section{Methods}
\label{sec:methods}


\subsection{Hybrid Neutrino}
\label{sec:hybrid}

Describe the method for implementing hybrid neutrinos:
particles that start as tracers, un-clustered. They sample
the Fermi-Dirac distribution with all velocities below some
initial slow value. Initially the analytic code follow the full evolution,
but then at some redshift the neutrino particles start to gravitate.

After the neutrino particles start to gravitate we modify the
analytic module so that neutrino perturbations are included only above a
minimum momentum corresponding to the initial slow value sampled by the
particle velocity distribution. In other words (insert YAH analytics here).

We attempted to include neutrinos dynamically at a late time, but
accurately modelling the non-linear distortions to the analytic
Fermi-Dirac function proved overly complex. As the extra particles
do not gravitate initially, their addition at early times does
not significantly alter the efficiency of the code.

\subsection{Initial Conditions}
\label{sec:initcond}
Describe improvements to the IC generation since the last paper.
These are: we solve the growth function analytically, rather than
using the $\Omega_m^{0.6}$ approximation, and include radiation
and neutrinos in the derived growth function.
We account for radiation and neutrinos in the Hubble function
that appears as a factor between the displacements and velocities
in the Zel'dovich approximation. We added the ability to account for
separate transfer functions with a separate collisionless fluid,
which has initial conditions set by the baryonic transfer function.


\subsection{Code improvements}
\label{sec:code}

We have altered our neutrino integrator to be a stand-alone module, largely
independent of the underlying N-body code. It should thus be easy to include
into codes other than Gadget. We have also provided a script to integrate into the
public version of Gadget-2, as well as the highly scalable public MP-Gadget
used for the BlueTides simulation. Our code is publically downloadable from.
We have added comprehensive unit tests and documentation.

\subsection{Simulations}
\label{sec:simulations}

All our simulations were run with MP-Gadget, using the scripts available
and the initial condition generator S-GenIC.

Simulations:
Mnu = 0  (512, 512), (200, 512), (200, 1024).
[OmegaFakeBaryon vs combined CDM+baryon with Mnu = 0.]

Mnu = 0.1: analytic.
(512, 512), (200, 512)
Mnu = 0.2: analytic, particle, hybrid.
(512, 512), (200, 512).
Mnu = 0.4: analytic, particle, hybrid.
(512, 512), (200, 512), (200,512,1024 *neutrinos*), (200, 2x1024).

Checks (all Mnu=0.4 (?), hybrid, (200,512)):
Varying vcrit from 500 to 300
Varying NuPartTime from 0.333 to 0.5
(run these first)


\section{Results}
\label{sec:results}

\begin{figure}
  \caption{Projected density plot of CDM and neutrinos from the hybrid simulation (compared with the particle?)}
  \label{fig:density_plot}
\end{figure}

\begin{figure}
  \caption{Plot showing the matter power spectrum ratios between analytic and hybrid. }
  \label{fig:matter_power}
\end{figure}

\begin{figure}
  \caption{Plot showing the neutrino power spectrum ratios between particle, analytic and hybrid.
  MONEY PLOT.}
  \label{fig:neutrino_power}
\end{figure}

\begin{figure}
  \caption{Plot showing the cross-correlation coefficient between neutrinos
  and DM in particle and hybrid code, checking approximation made in analytic method.}
  \label{fig:cross-corr}
\end{figure}

Perhaps follow this with a plot showing the cross-correlation coefficient between neutrinos+DM
in different simulations, to show that the structure is really the same in all of them.

\subsection{Numerical Checks}
\begin{figure}
  \caption{Plot showing the power spectrum for analytic compared to linear theory,
  showing that we reproduce (ie, our ICs are very very good). }
  \label{fig:linear_power}
\end{figure}

\begin{figure}
  \caption{Plot showing the difference in the neutrino power spectrum from changing NuPartTime is small. }
  \label{fig:nuparttime}
\end{figure}

\begin{figure}
  \caption{Plot showing the difference in the neutrino power spectrum from changing vcrit is small. }
  \label{fig:vcrit}
\end{figure}

\section{Conclusions}
\label{sec:conclusion}

This problem is comprehensively solved.

\section*{Acknowledgements}

\appendix

\section{Kspace Neutrino Manual}
\label{sec:manual}

Here describe the parameters of the kspace neutrinos,
as well as the installation procedure for Gadget 2.

\label{lastpage}

\bibliography{neutrinos}

\end{document}
